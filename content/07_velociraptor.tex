%! Licence = CC BY-NC-SA 4.0

%! Author = gianfluetsch
%! Date = 30. Dez 2021
%! Project = cydef_summary

\section{Velociraptor}
Velociraptor is a free and open-source software project developed by the Velocidex Company. Velociraptor is generally based on GRR, OSQuery, and Google's Rekall tools. Velociraptor allows users to collect Forensics Evidence, Threat Hunting, Monitoring artifacts, Executing remote triage process. As an open-source platform, Velociraptor continues to improve and evolve through inputs and feedback of digital forensics investigation and cybersecurity practitioner

Velociraptor was released in 2019 and allows for hunting across many thousands of machines. Inspired by OSQuery, Velociraptor implements a new query language dubbed \textit{VQL} (Velociraptor Query Language) which is similar to SQL but extends the query language in a more powerful way. Velociraptor also emphasizes ease of installation and very low latency — typically collecting artifacts from thousands of endpoints in a matter of seconds.

\subsection{Introduction}

\subsubsection{Registry Search}
Velociraptor can be used to search the registry of a client in several ways
\begin{enumerate}
    \item Manually look through the registry
    \item Design your own VQL query and use the Notebook
    \item Run a hunt on only Forensic to get the information
\end{enumerate}

\subsubsection{Notebook - raw\_reg}
Like the \textit{raw\_reg} accessor, the zip accessor also requires a url. Here is what a query for all files ending with .jpg in the zip file \lstinline|C:\Users\Bob\Desktop\images.zip| could look like:

\begin{lstlisting}[language=bash]
    SELECT * FROM glob(globs=url(scheme='ntfs', path='C:/Users/Bob/Desktop/images.zip', fragment='/**/*.jpg').String, accessor='zip')
\end{lstlisting}

\subsubsection{Notebook - foreach}
To not only do this for the current user but all users, you have to use a \textit{foreach} plugin.

The following statement would give you the Name and executable path for all executables that are running as the user(s) with the SID that is also running a process that has velociraptor in its executable path.

\begin{lstlisting}[language=bash]
    SELECT * FROM foreach(
    row={SELECT OwnerSid AS velociraptor_owner FROM pslist() WHERE Exe=~'velociraptor'},
    query={SELECT Name, Exe FROM pslist() WHERE OwnerSid=velociraptor_owner}
)
\end{lstlisting}

\subsubsection{Builtin Artifacts}
Velociraptor already comes with a number of pre-built artifacts for frequently needed queries.
This includes for example querying the registry (\lstinline|Windows.Registry.NTUser|).

\subsubsection{YARA Artifact}
You might have to use a foreach loop again where the rows statement gives you the PIDs and names of all processes and then scan for the YARA signature in the query block.

\newpage

\subsection{Lateral Movement}
Lateral movement means to move within the internal network to access the organization's target data and to exfiltrate the data. In this challenge you will solve tasks to detect lateral movement using Velociraptor.

\subsubsection{Detection PsExec}
To Detect if someone used \textit{PsExec} after passing the hash, there are multiple ways. One way is to design your own Artifact based on \lstinline|Windows.Registry.Sysinternals.Eulacheck| Artifact.

\subsubsection{Use Velociraptor Artifact}
In Velociraptor, use the Artifact \lstinline|Windows.EventLogs.AlternateLogon|. It will list all Windows events with ID \textit{4648}. Use the timestamp function from Velociraptor to change the timestamp to date time format.

\subsubsection{Event Correlation}
With the event results from the Artifacts used, do an event correlation by matching the time at which both events occurred. Both event should have a temporal correlation.

\subsubsection{Evidence on Destination}
As you will see in the result of Artifact \lstinline|Windows.EventLogs.AlternateLogon|, the TargetServerName was DC1. Look for evidence of execution on destination DC1 by searching for event ID \textit{4624} as mentioned in SANS Hunt Evil Poster.

\subsubsection{Detection Mimikatz}
Now you have collected some evidence that someone did misuse \textit{PsExec} for passing the hash. But to dump the hash from a node, you need a tool. One such tool is Mimikatz. Use the \lstinline|Windows.System.Amcache| Artifact to detect if Mimikatz was executed on the system.